\section{GIỚI THIỆU}

\hspace{2em} Ngành dược phẩm tại Việt Nam đang trải qua giai đoạn tăng trưởng mạnh mẽ với nhu cầu thông tin sản phẩm liên tục tăng cao. Tuy nhiên, các vấn đề hiện tại bao gồm: thiếu hụt nhân lực tư vấn dược để đáp ứng nhu cầu khách hàng, khó khăn trong cung cấp thông tin sản phẩm một cách nhất quán và kịp thời, cũng như chi phí cao trong duy trì các hệ thống hỗ trợ khách hàng truyền thống. Theo các nghiên cứu gần đây \cite{ChatGPT2022, chatbot2021}, sự xuất hiện của các chatbot AI hiện đại đã chứng minh khả năng cách mạng hóa cách thức tương tác giữa doanh nghiệp và khách hàng, với khả năng giảm thời gian phản hồi từ vài giờ xuống còn vài giây và tăng độ chính xác của thông tin lên trên 95\%.

\hspace{2em} Dự án này tập trung vào xây dựng một chatbot hỗ trợ bán thuốc bằng cách tích hợp các công nghệ tiên tiến như mô hình ngôn ngữ lớn (Large Language Models - LLMs) \cite{GPT32020, GPT42023, BERT2018}, mô hình nhúng văn bản (embeddings) \cite{SBERT2019, mBERT2021}, và kỹ thuật Truy xuất - Tăng cường (Retrieval-Augmented Generation - RAG) \cite{RAG2020}. Các công nghệ này cho phép hệ thống truy xuất chính xác thông tin sản phẩm từ cơ sở dữ liệu dược phẩm và sinh ra các câu trả lời rõ ràng, đầy đủ, phù hợp với nhu cầu cụ thể của khách hàng. Hơn nữa, các kỹ thuật nâng cao như Rerank RAG \cite{rerank2019, rerank2023} và Query Reformulation \cite{rephrase2023} có thể được áp dụng để tăng cường chất lượng truy xuất và sinh văn bản.

\hspace{2em} Mục tiêu chính của dự án là phát triển một hệ thống chatbot thông minh có khả năng: (1) truy xuất chính xác thông tin sản phẩm từ cơ sở dữ liệu dược phẩm; (2) sinh ra các câu trả lời rõ ràng và phù hợp với nhu cầu khách hàng; (3) hạn chế hiện tượng sinh ra thông tin giả mạo (hallucination) \cite{hallucination2021} thông qua cơ chế RAG. Phạm vi của dự án bao gồm xây dựng hệ thống từ đầu, tích hợp các công nghệ hiện đại, huấn luyện và đánh giá trên tập dữ liệu thực tế về sản phẩm dược phẩm.