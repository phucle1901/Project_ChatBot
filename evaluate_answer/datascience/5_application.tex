\section{TRIỂN KHAI}

Hệ thống được thiết kế theo kiến trúc Agent-based RAG (Retrieval-Augmented Generation) với quy trình xử lý truy vấn thông minh. Sơ đồ kiến trúc chi tiết như sau:

\begin{center}
	\includegraphics[width=\textwidth]{images/deploy.jpg}
\end{center}

\subsection{Thành phần chính}

\subsubsection{Module Crawling (Thu thập và xử lý dữ liệu)}
Phần này chịu trách nhiệm chuẩn bị nguồn dữ liệu cho hệ thống, bao gồm ba bước chính:
\begin{itemize}
	\item \textbf{Crawl Web}: Sử dụng các công cụ web scraping để thu thập dữ liệu từ các nguồn y tế trực tuyến, bao gồm thông tin về thuốc, bệnh, và các bài viết y tế chuyên môn.
	\item \textbf{Chunk + Embed}: Chia nhỏ tài liệu thu thập được thành các đoạn nhỏ (chunks) phù hợp với độ dài ngữ cảnh của mô hình. Sau đó, sử dụng embedding models (ví dụ như BERT, Sentence Transformers) để chuyển đổi các chunks này thành vector đặc trưng trong không gian vector cao chiều.
	\item \textbf{RAG Database}: Lưu trữ các vector embeddings cùng với metadata của tài liệu gốc trong một cơ sở dữ liệu vector (vector database) như Pinecone, Weaviate, hoặc Milvus. Cơ sở dữ liệu này cho phép truy xuất nhanh chóng các tài liệu liên quan dựa trên similarity search.
\end{itemize}

\subsubsection{Module Router (Định tuyến thông minh)}
Router là thành phần đầu tiên nhận truy vấn từ người dùng và quyết định hướng xử lý:
\begin{itemize}
	\item \textbf{User Query}: Người dùng nhập một câu hỏi hoặc truy vấn tự nhiên liên quan đến thông tin y tế.
	\item Phân loại truy vấn thành hai loại chính:
	\begin{enumerate}
		\item \textbf{Drug Information}: Truy vấn liên quan đến thông tin thuốc - được định tuyến đến luồng xử lý Split Query.
		\item \textbf{Store Database}: Truy vấn cần truy cập cơ sở dữ liệu cấu trúc (ví dụ như dữ liệu bệnh nhân, tiểu sử y tế) - được định tuyến đến Database + Answer.
	\end{enumerate}
\end{itemize}

\subsubsection{Luồng xử lý Drug Information}
Đối với các truy vấn liên quan đến thông tin thuốc, hệ thống thực hiện quy trình sau:

\begin{itemize}
	\item \textbf{Split Query}: Hệ thống phân tích và chia nhỏ truy vấn phức tạp thành các truy vấn con đơn giản hơn, giúp tăng độ chính xác và che phủ toàn diện các khía cạnh của câu hỏi gốc.
	\item \textbf{K Queries}: Tạo ra tập hợp các truy vấn con (từ 1 đến K truy vấn) tùy thuộc vào độ phức tạp của câu hỏi gốc.
	\item \textbf{RAG + Answer (lần 1)}: 
	\begin{itemize}
		\item Sử dụng similarity search trên RAG database để tìm kiếm các tài liệu hoặc đoạn văn bản liên quan nhất (top-K results).
		\item Truyền các tài liệu tìm được cùng với truy vấn đến LLM để sinh ra câu trả lời được hỗ trợ bằng dữ liệu.
	\end{itemize}
	\item \textbf{RAG + Answer (lần 2)}: 
	\begin{itemize}
		\item Đánh giá chất lượng câu trả lời từ bước trước.
		\item Nếu câu trả lời \textbf{unsatisfied} (không đạt yêu cầu): Thực hiện một lần tìm kiếm và sinh câu trả lời khác với các tham số điều chỉnh.
		\item Nếu vẫn không đạt yêu cầu: Chuyển sang Web Search + Answer.
	\end{itemize}
	\item \textbf{Web Search + Answer}:
	\begin{itemize}
		\item Khi câu trả lời từ RAG database không đủ thỏa mãn, hệ thống tìm kiếm thông tin từ Internet.
		\item Sử dụng các search engine để thu thập thông tin bổ sung từ các nguồn trực tuyến.
		\item Kết hợp kết quả từ web search để tạo ra câu trả lời toàn diện hơn.
	\end{itemize}
	\item \textbf{Satisfied}: Khi câu trả lời đạt yêu cầu (từ RAG hoặc Web Search), chuyển đến Final Answer.
\end{itemize}

\subsubsection{Luồng xử lý Store Database}
Đối với các truy vấn cần truy cập dữ liệu có cấu trúc:
\begin{itemize}
	\item \textbf{Database Tools}: Sử dụng SQL hoặc các công cụ truy vấn cơ sở dữ liệu để tương tác với dữ liệu.
	\item \textbf{Database + Answer}:
	\begin{itemize}
		\item Trích xuất thông tin từ các bảng dữ liệu có cấu trúc.
		\item Kết hợp kết quả cơ sở dữ liệu với xử lý ngôn ngữ tự nhiên để tạo ra câu trả lời dễ hiểu.
		\item Chuyển trực tiếp đến Final Answer.
	\end{itemize}
\end{itemize}

\subsubsection{Final Answer (Câu trả lời cuối cùng)}
Kết quả cuối cùng được trả về cho người dùng, bao gồm:
\begin{itemize}
	\item Câu trả lời chính để trả lời câu hỏi của người dùng.
	\item Các tham chiếu (references) chỉ ra nguồn dữ liệu cho mỗi phần của câu trả lời (từ RAG database, Web search, hoặc Store database).
	\item Độ tự tin (confidence score) để người dùng biết mức độ chắc chắn của hệ thống.
\end{itemize}

\subsection{Luồng xử lý tổng thể}

Hệ thống hoạt động theo luồng sau:
\begin{enumerate}
	\item Người dùng nhập truy vấn tự nhiên (User Query).
	\item Router phân loại truy vấn:
	\begin{itemize}
		\item Nếu là Drug Information: Chuyển sang luồng Split Query → K Queries → RAG + Answer.
		\item Nếu là Store Database: Chuyển sang luồng Database Tools → Database + Answer.
	\end{itemize}
	\item Đối với luồng Drug Information:
	\begin{itemize}
		\item Thực hiện RAG + Answer lần 1.
		\item Nếu unsatisfied: Thực hiện RAG + Answer lần 2.
		\item Nếu vẫn unsatisfied: Thực hiện Web Search + Answer.
		\item Khi satisfied: Chuyển đến Final Answer.
	\end{itemize}
	\item Đối với luồng Store Database: Kết quả từ Database + Answer chuyển trực tiếp đến Final Answer.
	\item Trả về Final Answer cho người dùng.
\end{enumerate}

\subsection{Ưu điểm của kiến trúc}
\begin{itemize}
	\item \textbf{Định tuyến thông minh}: Router giúp phân loại truy vấn chính xác và điều hướng đến nguồn dữ liệu phù hợp nhất.
	\item \textbf{Cơ chế đảm bảo chất lượng}: Với luồng RAG + Answer hai lần và Web Search backup, hệ thống đảm bảo cung cấp câu trả lời tốt nhất có thể.
	\item \textbf{Tích hợp đa nguồn}: Kết hợp cả dữ liệu phi cấu trúc (RAG database, Web) và dữ liệu có cấu trúc (Store database).
	\item \textbf{Khả năng mở rộng}: Kiến trúc module cho phép dễ dàng thêm các nguồn dữ liệu hoặc cải thiện từng thành phần độc lập.
\end{itemize}